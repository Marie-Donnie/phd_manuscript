\markboth{}{}
% Plus petite marge du bas pour la quatrième de couverture
% Shorter bottom margin for the back cover
\newgeometry{inner=30mm,outer=20mm,top=40mm,bottom=20mm}

%insertion de l'image de fond du dos (resume)
%background image for resume (back)
\backcoverheader

% Switch font style to back cover style
\selectfontbackcover{ % Font style change is limited to this page using braces, just in case

\titleFR{Cheops, une approche externe pour géo-distribuer en périphérie les applications à base de micro-services.}

\keywordsFR{Informatique nuagique, Informatique périphérique, modularité, maillage de services}

\abstractFR{Le passage de l'informatique en nuage à l'informatique en
périphérie a modifié les exigences relatives aux applications qui y
sont exécutées.
%
Si les applications actuelles de l'informatique en nuage sont
extrêmement robustes dans ce contexte, elles n'ont pas été
conçues pour faire face aux défis inhérents à l'informatique en
périphérie, en particulier les déconnexions et les latences élevées
que l'on peut observer entre des sites éloignés.
%
Puisque nous disposons déjà d'applications pour le nuage robustes et
au code volumineux, la question qui se pose est la suivante
: serait-il possible de les utiliser en périphérie en gérant
l'échelle et la distribution géographique ?
%
Pour répondre à cette question, je présente d'abord différentes
approches existantes pour faire des applications fonctionnant en
périphérie et les lacunes de ces solutions, tout en
conservant les réponses intéressantes à des problèmes spécifiques.
%
A partir de cette étude, je présente la solution construite pour
amener les applications du nuage à la périphérie tout en donnant aux
utilisateurices le choix du lieu d'exécution de leurs requêtes.
%
Cette solution s'appuie sur la modularité des applications existantes
du nuage pour créer une approche ressemblant à un maillage de services
qui intercepte les demandes entre les services et les redirige en
fonction du langage spécifique à un domaine (DSL) que nous avons créé
pour permettre aux utilisateurices de spécifier des collaborations
entre plusieurs sites en périphérie.
%
% Elle cherche à répondre aux défis de l'informatique en périphérie,
% notamment les les déconnexions, en exécutant les demandes localement
% si les utilisateurs ne spécifient d'autre localisation via le DSL.
}



\titleEN{Cheops, a service mesh to geo-distribute micro-service applications at the Edge.}

\keywordsEN{Cloud Computing, Edge Computing, modularity, service mesh}

\abstractEN{The shift from Cloud Computing to Edge computing has
changed the requirements for the applications running there.
%
While the current Cloud computing applications are extremely robust in
the context of the Cloud, they were not made to face the challenges
inherent to the Edge computing, especially disconnections and high
latencies that can be observed between far sites.
%
Since we already have robust and huge Cloud applications, the question
that remains is: could it be possible to use them at the Edge by
managing the scale and the geo-distribution of the infrastructure
outside of the business logic?
%
To answer this question, I study different existing approaches to
bring applications to the Edge and identify what is missing in these
solutions, as well as keeping track of the interesting answers to
specific problems.
%
From this study, I present the solution built to bring Cloud
applications at the Edge while giving users the choice of the location
for their requests executions.
%
This solution relies on the modularity of these existing Cloud
applications to create a service-mesh like approach that intercepts
the requests between services and redirects them according to the
domain-specific language (DSL) we created that allows users to specify
collaborations between different Edge sites.
%
% It is aimed at coping with the Edge challenges, especially
% disconnections, by executing requests locally as much as possible when
% the users do not specify differently using the DSL.

}

}

% Rétablit les marges d'origines
% Restore original margin settings
\restoregeometry
