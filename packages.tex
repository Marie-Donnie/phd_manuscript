% Remplir les metadonnees du pdf
% Fill the pdf metadata
\hypersetup{
    pdfauthor   = {Marie Delavergne},
%    pdftitle    = {Th\`{e}se de doctorat de XYZ},
%    pdfsubject  = {Th\`{e}se de doctorat de XYZ},
%    pdfkeywords = {mots-cl\'{e}s},
}

\geometry{vmargin=4.0cm}

\usepackage[utf8]{inputenc}

\usepackage{amsmath,amssymb,amsfonts}


\usepackage{algorithmic}


\usepackage{graphicx}


\usepackage{textcomp}


\usepackage[dvipsnames]{xcolor}


\usepackage{hyperref}
%\renewcommand\UrlFont{\color{MidnightBlue}\rmfamily}
\hypersetup{colorlinks=true, citecolor=RoyalPurple, filecolor=black, linkcolor=BlueViolet, urlcolor=MidnightBlue}

\addto\extrasenglish{
    \def\chapterautorefname{Chapter}
}

\addto\extrasenglish{
    \def\sectionautorefname{Section}
}

\usepackage{xspace}

%% Code highlight
\usepackage{upquote}  % To set `upquote=true` in `lstset`

\usepackage{listings}

\usepackage{enumitem}

\usepackage{verbatim}

\usepackage{wrapfig}
\usepackage{sidecap}
\usepackage{subcaption}

%\usepackage{caption}

%% For background highlight in listing
%% See https://tex.stackexchange.com/a/159602
\newcommand{\hlight}[1]{%
  \colorbox{cyan!10}{#1}%
}

% Listings preferences, with upquote => if we don't use upquote, no need for textcomp either
\lstset{
  basicstyle=\ttfamily\footnotesize,
  upquote=true,
  morecomment=[l]{\#},
  commentstyle=\ttfamily\itshape,
  numbersep=3pt,numberstyle=\ttfamily\tiny,
}



%%   ___          ___
%%  (o o)        (o o)
%% (  V  ) TIKZ (  V  )
%% --m-m----------m-m--

\usepackage{tikz}
\usetikzlibrary{
  positioning, % syntax `below=<optional length> of ...`
  backgrounds, % pgfonlayer{background}
  arrows.meta, % for more arrows options
  matrix,
  fit,
  calc,        % Math operations on coordinate `($s_i+(0,-1mm)$)`
  shapes.geometric,
  shapes.symbols,
}
\usepackage{etoolbox}  % For foreach in matrix

%% Color blind accessible palette
\definecolor{myyellow}{RGB}{255,194,10}
\definecolor{myblue}{RGB}{12,123,220}

%% Tikz Shapes
\tikzset{every picture/.style={node distance=5mm}}
\tikzstyle{app}=[rounded corners, thin, fill=black]
\tikzstyle{labeled}=[minimum size=3mm, inner sep=0pt]
\tikzstyle{endpoint}=[labeled, circle, draw, thin, labeled, minimum size=4mm]
\tikzstyle{client}=[endpoint, fill=black, minimum size=2mm]
\tikzstyle{service}=[matrix of math nodes, draw, row sep=2mm]
\tikzstyle{db}=[cylinder, shape border rotate=90, draw]

% Macro to draw a service with many endpoints and a specific id
% `\service{id}{name}{endpoint_1,endpoint_2,ep_3}'
\newcommand\idservice[4][]{
  %% Build the matrix content outside
  %% https://tex.stackexchange.com/a/22276
  %% Use also \show\mymatrixcontent to see the expansion
  \let\mymatrixcontent\empty
  \expandafter\gappto\expandafter\mymatrixcontent\expandafter{|[draw=none, minimum size=0mm]| #3\\}%
  \foreach \endpoint in {#4} {%
    \expandafter\gappto\expandafter\mymatrixcontent\expandafter{|[endpoint]|}
    \expandafter\gappto\expandafter\mymatrixcontent\expandafter{\endpoint\\}
  }

  \matrix (#2) [service,#1] {\mymatrixcontent};
}

% Service with same name and id,
% `\service{name}{endpoint_1,endpoint_2,ep_3}'
\newcommand\service[3][]{\idservice[#1]{#2}{#2}{#3}}
% Service mesh with id, service mesh
\newcommand\idmesh[4][]{\idservice[dashed,draw=none,#1]{#2}{#3}{#4}}
\newcommand\mesh[3][]{\idmesh[#1]{#2}{#2}{#3}}

%%   ___              ___
%%  (o o)            (o o)
%% (  V  ) END_TIKZ (  V  )
%% --m-m--------------m-m--

%% Hyphenation
%% TeX won't hyphenate a word that's already been hyphenated. For
%% example, "geo-distributed". Replace the hyphen in the
%% "geo-distributed" with a \hyph command `geo\hyph{}distributed`.
%% See https://www.texfaq.org/FAQ-nohyph
\def\hyph{-\penalty0\hskip0pt\relax}
\hyphenation{geo-dis-tri-bu-tion geo-dis-tri-bu-ted Dev-Ops}


\usepackage[textwidth=17mm, colorinlistoftodos]{todonotes}


\definecolor{CbBlue}{HTML}{332288}
\definecolor{CbGreen}{HTML}{117733}
\definecolor{CbTeal}{HTML}{44aa99}
\definecolor{CbCyan}{HTML}{88ccee}
\definecolor{CbOrange}{HTML}{ffab40}
\definecolor{CbRose}{HTML}{cc6677}
\definecolor{CbFuschia}{HTML}{aa4499}
\definecolor{CbPlum}{HTML}{882255}

\newcommand{\tododo}[1]{\todo[inline, shadow, color=CbTeal!80]{#1}}
\newcommand{\todoref}[1]{\todo[inline, shadow, color=SeaGreen!80]{#1}}
\newcommand{\todomore}[1]{\todo[inline, shadow, color=RoyalPurple!40]{#1}}
\newcommand{\todocheck}[1]{\todo[inline, shadow, color=Yellow!40]{#1}}
\newcommand{\todono}[1]{\todo[inline, shadow, color=OrangeRed!80]{#1}}
\newcommand{\todofig}[1]{\missingfigure[figcolor=white]{#1}}


% colored cite brackets
\DeclareOuterCiteDelims{cite}{\textcolor{RoyalPurple}{\bibopenbracket}}{\textcolor{RoyalPurple}{\bibclosebracket}}


\usepackage{epigraph}
%\usepackage{epipart}

%\usepackage[printonlyused,withpage]{acronym}

\usepackage[acronym,toc]{glossaries}
% https://tex.stackexchange.com/questions/198231/remove-page-break-before-glossary
\renewcommand*{\glsclearpage}{}

\newglossarystyle{better}{%
  \glossarystyle{list}%
  \renewcommand*{\glossaryentryfield}[5]{%
    \item[\glsentryitem{##1}\glstarget{##1}{##2}] ##3%
      }%
  \renewcommand*{\glsgroupskip}{}%
}

\glossarystyle{better}


\newcommand{\eg}{{e.g.,}\xspace}
\newcommand{\ie}{{i.e.,}\xspace}
\newcommand{\scl}{scope-lang\xspace}
\newcommand{\os}{OpenStack\xspace}
\newcommand{\ks}{Kubernetes\xspace}
\newcommand{\sOne}{Site~1\xspace}
\newcommand{\sTwo}{Site~2\xspace}
\newcommand{\collab}{collaboration}
\newcommand{\mscl}{\mathit{Scope}}
\newcommand{\mscll}[1]{\mathit{Scope}=\mathit{#1}}
\newcommand{\dc}{\gls{dc}\xspace}
\newcommand{\dcs}{\glspl{dc}\xspace}
\newcommand{\Dc}{Data center\xspace}
\newcommand{\Dcs}{Data centers\xspace}


\usepackage{multirow}

\usepackage{tcolorbox}
